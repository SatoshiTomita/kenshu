% 日本語レポート雛形(LuaLaTeX / LuaTeX-ja)
\documentclass[12pt,a4paper]{ltjsarticle}

% パッケージ
\usepackage{graphicx}
\usepackage{hyperref}
\usepackage{amsmath,amssymb}
\usepackage{booktabs}

% メタ情報
\title{レポートタイトル}
\author{著者名}
\date{\today}

\begin{document}
\maketitle

\begin{abstract}
  ここに要旨を入力します。
\end{abstract}

\section{はじめに}
背景と目的を書きます。

\section{方法}
実験や解析の手法を記述します。

\section{結果}
結果を示します。図や表は以下のように挿入します。

\begin{figure}[htbp]
  \centering
  \includegraphics[width=0.6\linewidth]{example-image}
  \caption{例の図}
  \label{fig:example}
\end{figure}

\section{考察}
結果の解釈や今後の課題を書きます。

\section{結論}
結論を簡潔にまとめます。

\begin{thebibliography}{9}
  \bibitem{latex} Leslie Lamport. \textit{LaTeX: A Document Preparation System}. 2nd ed., 1994.
\end{thebibliography}

\end{document}
